\section{Future Work}\label{sec:futurework}


\emph{Security Extensions:}

A virtual observatory will enable researchers to simultaneously discover and acquire data streams from multiple partnering observatories.  Navigating the network of resources and accessing sensitive data can be a laborious task, requiring knowledge of local protocols and site-level access-control procedures.  
% System security is paramount, but usability and performance must be maintained.  
To that end, a certificate-based single sign-on protocol will provide robust, but manageable, cyber-security.  RBNB DataTurbine currently has basic support for username/password access control, and can also regulate which IP addresses have access to a RBNB DataTurbine system in a manner similar to tcpwrappers ~\cite{tcpwrappers}.
% However, all data is sent in the clear and is subject to sniffing and similar vulnerabilities. Our deployments call for employing the RBNB DataTurbine for command and control of instruments in addition to data acquisition.  In both cases, the sensitivity and integrity of network communications must be enforced.  
Single sign-on across virtual organizations has been a research topic for several years and there are existing projects that can provide technical specification and tools~\cite{CAMERA, GEON}. The political and technical difficulties of security solutions across administrative domains have been well documented and are formidable. Propagating credentials across federated RBNB DataTurbines generates complex interaction scenarios and raises requirements for fine-grained access control.
% which has potential performance implications.
% In order to support secure transmission of data, we will use Java�s existing support for TLS/SSL to encrypt message channels. We will modify the RBNB DataTurbine code, both on the server-side and the client-side. 
To support user authentication and authorization, we will investigate and evaluate existing authentication/authorization systems, including GAMA~\cite{bhatia-05}, Shibboleth~\cite{Shibboleth}, and GridAuth~\cite{GridAuth}. For integration, we will make additional modifications to the RBNB DataTurbine server so that it calls an external service for authentication, authorization and access control. % As part of our security-related middleware improvements, we will utilize a dedicated Sun Fire T2000 server which has hardware cryptography accelerator cards (ref: Facilities Section). We will investigate the use of these hardware cryptography accelerator cards for transmission of encrypted data. Finally, we will create Web-based tutorials on how to enable, configure and use the new features. 

\emph{Extensions to support Interoperability and Federation of multiple observatories:} Interoperability across platforms and observing systems is a challenging requirement. To address this requirement, we propose a two-part solution. First, we will provide interoperability via metadata annotation of data streams and instruments according to emerging interoperability standards. Through Open Geospatial Consortium�s (OGC)~\cite{OGC} Sensor Web Enablement (SWE) initiative, several community XML standards are emerging for transmitting sensor data. Transducer Markup Language (TML) enables the interoperability of heterogeneous sensor systems by providing a self describing data exchange protocol based on XML~\cite{TML}.  Third-party applications can utilize this data-stream XML description to interpret and utilize RBNB DataTurbine streams. For describing instruments, we will employ Sensor Markup Language (SML), which is a second OGC standard that captures device properties ~\cite{SML}.  Once the appropriate encodings are developed, we will write a software adapter that will read data streams in the RBNB DataTurbine internal format and write output that is encoded in TML/SML. System developers and engineers can then use off-the-shelf XML parsers and processing tools to interpret and access RBNB DataTurbine data streams. This provides a basis for interoperability across disparate observing systems middleware via standards compliant access to real-time sensor data streams. Note that we are not addressing policy issues in operating across jurisdictional boundaries since this is outside the scope of this project.  Instead, we are providing the basic middleware interoperability services for self-describing data streams.

% We are working on extending RBNB DataTurbine to support federating observatories. 
% The federation of DataTurbine servers is therefore an area of active investigation. Issues arise when considering metadata, where standards are evolving rapidly. In concert with Creare (the authors of the DataTurbine), we have created metadata for channel units (e.g. '$m/s^2$, degrees Celsius') but many others have yet to be defined. In particular, the rise of geo-tagged data lends an urgency to defining metadata for positions, coordinate systems and measurement error.


% We are considering multiple approaches that partition the work differently between client and server. For example, one idea is to run a central server that has a list of DataTurbine instances. The central server would periodically poll the instances, query their channel lists and metadata and cache the result. On top of this, a portal would allow for user queries and returning results as JNLP files for smart clients, or as RBNB-formatted URLs. For example: \url{rbnb://dt.cleos.sdsc.edu/NEON/NWP-JR/BHTemp-2/}

\emph{Integration with Streaming databases:}
Recently there has been a considerable research in the area of streaming databases~\cite{chandrasekaran02streaming, chen00niagaracq, abadi03aurora, chandrasekaran03telegraphcq}. In future, we would like to explore integration of streaming databases on top of RBNB DataTurbine. In that case, RBNB will provide a reliable data acquisition and transport substrate and would stream tuples
into stream processing engine. 
